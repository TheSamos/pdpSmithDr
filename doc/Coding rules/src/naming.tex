\section{Naming conventions}

\subsection{Identifiers}
Use clear and informative names for identifiers (classes, members,
parameters, variables).\\
Clear and informative names are one of the best tools for creating
easily understandable code. The name of any identifier should
succinctly describe the purpose of that identifier.\\

Identifiers names must be a mixture of upper and lowercase letters to
delineate individual words.
\begin{itemize}
\item[$\bullet$] Each word in a class name must begin with a capital
  letter, e.g. {\bf class RandomProcess;}
\item[$\bullet$] In a function name, use upper case letters as word
  separators, and lower case for the rest of a word, e.g. {\bf void getDescription();}
\item[$\bullet$] For macros, enumeration constants and global
  constants, words are capitalized and separated by underscores,
  e.g. {\bf DATA\_VALID}, {\bf const int MIN\_WIDTH = 1}.
\item[$\bullet$] Typedef names use the same naming policy as
  for a class.
\end{itemize}
\ \\
Class names will generally consist of nouns or noun combinations.\\
Function names will generally begin with a verb as they describe actions.\\
\ \\
Method argument names follow the same rules as function names.\\
Class members should be prefixed with {\bf m\_} (letter {\it m}
followed by an undescore). The rest of the name is similar to method
argument names (including the beginning lower case letter).

\subsection{Class organization}
\begin{itemize}
\item[$\bullet$] In general there will be one class declaration per
  header file. In some cases, smaller related classes may be grouped
  into one header file.
\item[$\bullet$] public, protected and private sections in the class
  declaration should be ordered so that the public section comes
  first, then the protected section, and lastly the private section.
\item[$\bullet$] Within each section, member functions and member data
  should not be mixed.
\end{itemize}

\subsection{Files}
\label{file_extension}
A file must have the same name as the class it contains.
\begin{itemize}
\item[$\bullet$] C++ header file names should have the extension ".hpp".
\item[$\bullet$] C++ source file names should have the extension ".cpp".
\end{itemize}
\ \\ Use special ".hxx" extension for files containing the definition
of template classes or inline functions (see
sections~\ref{sec::templates}~and~\ref{sec::inline} for further
details).
