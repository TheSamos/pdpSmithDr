\section{Formatting}

\subsection{Indentation, spacing and trailing whitespaces}
Indentation must be achieved using spaces; do not use tabs.\\
Indent using 2 spaces for each level.\\

You should use one space after each keyword (if, for, while, ...), use
one space after a comma.\\

For example:
\begin{algorithm}[H]
\codeindent{1}for (int i=0; i<10; ++i) \{ \\
\codeindent{2}... \\
\codeindent{1}\} \\
 \\
\codeindent{1}if (a >= 0) \{ \\
\codeindent{2}... \\
\codeindent{1}\} else \{ \\
\codeindent{2}... \\
\codeindent{1}\}
\end{algorithm}

Avoid trailing whitespaces in source code, as it may cause unecessary
differences for files under version control.

\subsection{One statement per line}
Each line should contain at most one statement.\\

For example:
\begin{algorithm}[H]
\codeindent{1}++argv;                /* Correct */ \\
\codeindent{1}--argc;                /* Correct */ \\
\codeindent{1}++argv; --argc;        /* AVOID! */
\end{algorithm}

\subsection{Brackets}
Opening braces should be on the same line of the statement, except for
class and functions/methods declarations.\\
Closing braces are always on their own line.\\
It is recommended to add braces even if they contain a single line.\\
It can be a good idea to add a comment to closing braces identifying
the corresponding opening braces if these ones are not in the visible
part of your code editor.\\
\ \\
Do not put parentheses next to keywords; put a space between.\\
Do put parentheses next to function names.\\

For example:
\begin{algorithm}[H]
MyClass::MyClass(size\_t n) : m\_vector(n) \{ \\
\codeindent{1}for (int i=0; i<n; ++i) \{ \\
\codeindent{2}m\_vector[i] = i; \\
\codeindent{1}\} \\
\} \\
 \\
void \\
MyClass::f(int i) const \{ \\
\codeindent{1}assert(0 <= i); \\
\codeindent{1}assert(m\_vector.size() > i); \\
\codeindent{1}if (i == m\_vector[i]) \{ \\
\codeindent{2}std$::$cerr$<<$"ok !"$<<$std$::$endl; \\
\codeindent{1}\} else \{ \\
\codeindent{2}std$::$cout$<<$"ko !!!"$<<$std$::$endl; \\
\codeindent{1}\} \\
\}
\end{algorithm}

% instructions
\subsection{Instructions}
\subsubsection{If-then-else.}
Respect brackets rule.\\
Prefer variable first in condition format.\\

For example:
\begin{algorithm}[H]
\codeindent{1}if (a < 0) \{ \\
\codeindent{2}std$::$cerr$<<$"a is negative!"$<<$std$::$endl; \\
\codeindent{1}\} else if (a == 0)\{ \\
\codeindent{2}std$::$cerr$<<$"a is zero!"$<<$std$::$endl; \\
\codeindent{1}\} else \{ \\
\codeindent{2}std$::$cerr$<<$"a is positive!"$<<$std$::$endl; \\
\codeindent{1}\}
\end{algorithm}

\subsubsection{Switch.}
Falling through a case statement into the next case statement shall be
permitted as long as a comment is included.\\
The default case should always be present and trigger an error if it
should not be reached, yet is reached.\\
If you need to create variables, put all the code in a block.\\

For example:
\begin{algorithm}[H]
\codeindent{1}switch(...) \{ \\
\codeindent{1}case 1: \\
\codeindent{2}\{ \\
\codeindent{3}... \\
\codeindent{2}\} \\
\codeindent{2}break; \\
\codeindent{1}case 2: \\
\codeindent{2}... \\
\codeindent{1}// FALL THROUGH \\
\codeindent{1}case 3: \\
\codeindent{2}\{ \\
\codeindent{3}int a; \\
\codeindent{3}... \\
\codeindent{2}\} \\
\codeindent{2}break; \\
\codeindent{1}default: \\
\codeindent{2}\{ \\
\codeindent{3}assert(!"Should not get here!"); \\
\codeindent{2}\} \\
\codeindent{1}\}
\end{algorithm}

% members
\subsection{Members}
\subsubsection{Const elements.}
Use const at the beginning of the expression.\\
Prefer
\begin{algorithm}[H]
\codeindent{1}const char* p;
\end{algorithm}

to
\begin{algorithm}[H]
\codeindent{1}char const* p;
\end{algorithm}

\subsubsection{Pointers and references.}
The {\bf \&} and {\bf *} tokens should be adjacent to the type, not
the name.\\

For example:
\begin{algorithm}[H]
int* iPtr;\\
int\& iRef;
\end{algorithm}
