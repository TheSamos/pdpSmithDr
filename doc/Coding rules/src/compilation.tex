\section{Compilation}

\subsection{Choice of the compiler}
You program in {\bf C++}, and the primary compiler you use is {\bf
  gcc}. Try to test your code with different compilers. If you use
g++, try to test different versions (4.2, 4.5, 4.6). A good way to do
that is to compile on various linux distributions (you can use
virtualbox for example to easily install several distributions on your
machine). Try also to compile with different compilers if possible:
llvm-g++, clang, icpc, ...

\subsection{Compiler flags}
You should all use {\bf -Wall -Wextra} (and even {\bf -pedantic}) to
compile our code, and correct the code to avoid most warnings.\\ G++
also provides flags {\bf -Wshadow} and {\bf -Weffc++}, very
interesting for C++.\\

You should provide both a debug and an optimized version.\\ A variable
in the Makefile could be enough to decide if you compile a debug or
optimized version.\\ The optimized version should be the fastest
possible: use {\bf -O3} (or -O2) and {\bf -DNDEBUG} during
compilation.\\ Compile with {\bf -g} for the debug version.\\

Intermediary results are often interesting for publications or even
for debugging. It should be possible to extract meaningful
intermediary results from your program without too much effort. You
can for example add a command line parameter or define a specific
compilation flag.
