\section{Layout}

\subsection{Declaration - definition}
You must separate declaration from definition of your classes and
functions.\\
Declaration should be in a header file, définition should be in a
source file (see section~\ref{file_extension}).

% header files
\subsection{Header files}
\subsubsection{Include guards.}
\label{sec:include_guards}
Use include guards in header files to prevent multiple inclusion:
\begin{algorithm}[H]
\#ifndef NAMESPACE\_CLASSNAME\_HPP \\
\#define NAMESPACE\_CLASSNAME\_HPP \\
\codeindent{1}... \\
\#endif /* ! NAMESPACE\_CLASSNAME\_HPP */
\end{algorithm}
{\bf Important: } Use also include guards for ".hxx" files.

\subsubsection{Self-sufficient header files.}
Make your header files self-sufficient. That is to say they include
all needed dependencies and can be compiled standalone using the
following simple test:
\begin{algorithm}[H]
\#include <header.hpp> \\
 \\
int \\
main \{ \\
\codeindent{1}return 0; \\
\}
\end{algorithm}

\subsubsection{Avoid global variables.}
Class data members must always be private (or protected).\\
If access to them is required then this must be provided through
public or protected member functions.

% source files
\subsection{Source files}
\subsubsection{Member order.}
Public class members must be declared first, then protected members,
and lastly private members. Moreover member functions must be
separated from member variables.\\

For example:
\begin{algorithm}[H]
class A \{ \\
\\
public: \\
\codeindent{1}/* public member functions */ \\
 \\
protected: \\
\codeindent{1}/* protected member functions */ \\
 \\
private: \\
\codeindent{1}/* private member functions */ \\
 \\
public: \\
\codeindent{1}/* public member variables */ \\
 \\
protected: \\
\codeindent{1}/* protected member variables */ \\
 \\
private: \\
\codeindent{1}/* private member variables */ \\
 \\
\};
\end{algorithm}
{\bf Important note: } Public member variables should be strongly avoided!

\subsubsection{No magic numbers.}
Do not hard-code values inside your program. For example, do not
hard-code image width, or number of P frames between two I frames, or
anything else.\\
If you have to hard-code something, use a global const variable
(preferable to a define) so that you only have to change it once.\\

For example, replace:
\begin{algorithm}[H]
Matrix m(100, 100);
\end{algorithm}

by:
\begin{algorithm}[H]
static const DEFAULT\_MATRIX\_NBROWS = 100; \\
static const DEFAULT\_MATRIX\_NBCOLS = 100; \\
Matrix m(DEFAULT\_MATRIX\_NBROWS, DEFAULT\_MATRIX\_NBCOLS);
\end{algorithm}

\subsubsection{Use short functions.}
Functions should be at most one page of the text editor.\\
A function that spreads on more than one page should be subdivided in
smaller functions.

% required members
\subsection{Required members}
\subsubsection{Default constructor and the rule of three.}
The {\bf default constructor}, {\bf copy constructor}, {\bf assignment
  operator}, and (virtual) {\bf destructor} should be either explicitly declared
or made inaccessible and undefined rather than relying on the
compiler-generated defaults.\\
The flag {\bf -Weffc++} for g++ can help you to find such cases.\\

For example, replace:
\begin{algorithm}[H]
class MyClass \{ \\
 \\
public: \\
\codeindent{1}MyClass() \{ \\
\codeindent{2}m\_mat = new Matrix; \\
\codeindent{1}\} \\
 \\
protected: \\
\codeindent{1}Matrix* m\_mat; \\
 \\
\};
\end{algorithm}

by:
\begin{algorithm}[H]
class MyClass \{ \\
 \\
public: \\
\codeindent{1}MyClass() \{ \\
\codeindent{2}m\_mat = new Matrix; \\
\codeindent{1}\} \\
 \\
\codeindent{1}~MyClass() \{ \\
\codeindent{2}delete m\_mat; \\
\codeindent{1}\} \\
 \\
private: \\
\codeindent{1}MyClass(const MyClass\&); \\
\codeindent{1}MyClass\& operator=(const MyClass\&); \\
 \\
protected: \\
\codeindent{1}Matrix* m\_mat; \\
 \\
\};
\end{algorithm}

Here, there is no need to implement copy constructor and assignment operator.
If using C++0x, you should even write;
\begin{algorithm}[H]
private: \\
\codeindent{1}MyClass(const MyClass\&) = delete; \\
\codeindent{1}MyClass\& operator=(const MyClass\&) = delete;
\end{algorithm}


\subsubsection{Virtual destructor.}
A destructor of a class that will be inherited must be declared {\emph virtual}.

% others
\subsection{Namespaces}
Namespaces must be unique! Make your namespace verbose, just like a
absolute path. Prefix with your application name.\\
Use lower case letters.\\

For example:
\begin{algorithm}[H]
namespace myapp \{ \\
 \\
\codeindent{1}namespace core \{ \\
 \\
\codeindent{2}... \\
\codeindent{1}\} \\
 \\
\} \\
 \\
myapp$::$core$::$Image img;
\end{algorithm}

{\bf NEVER add a {\it using namespace} directive to header
files!} If a header file (.hpp) contains a \emph{using namespace} (such as
\emph{using namespace std;}) directive, every file that includes this
file will use the namespace and so it removes all the interest of the
namespace.\\
It may be a good idea to never write such a directive at all.\\

You can use namespace shortcuts for convenience.\\

For example:
\begin{algorithm}[H]
namespace alias = a$::$very$::$long$::$namespace; \\
 \\
class YourClass : public alias$::$TheirClass \{ \\
\codeindent{1}... \\
\};
\end{algorithm}

\subsection{Templates}
\label{sec::templates}
Most of compilers impose that templates must be entirely in a single
header file. To presere our declaration/definition separation, you
must use the following tip.\\Template declaration must be in a header
file (".hpp" extension); template definition must be in a specific
source file (".hxx" extension). The header file must include the
source file; this inclusion must occur \emph{at the end} of the header
file. \emph{Use include guards} (section~\ref{sec:include_guards}) in
both header and source files!\\ For example, a Vector.hpp file is the
following:
\begin{algorithm}[H]
\#ifndef MYAPP\_VECTOR\_HPP \\
\#define MYAPP\_VECTOR\_HPP \\
 \\
namespace myapp \{\\
 \\
\codeindent{1}template <typename T> \\
\codeindent{1}class Vector \{ \\
 \\
\codeindent{1}public: \\
\codeindent{2}Vector(); \\
\codeindent{2}... \\
\codeindent{2}bool isEmpty() const; \\
 \\
\codeindent{1}\}; // end class declaration \\
 \\
\} \\
 \\
\#include "Vector.hxx" \\
 \\
\#endif /* ! MYAPP\_VECTOR\_HPP */
\end{algorithm}

and the corresponding Vector.hxx file is:
\begin{algorithm}[H]
\#ifndef MYAPP\_VECTOR\_HXX \\
\#define MYAPP\_VECTOR\_HXX \\
 \\
template <typename T> \\
myapp::Vector<T>$::$Vector() \{ \\
\codeindent{2}... \\
\} \\
 \\
template <typename T> \\
bool \\
myapp::Vector<T>$::$isEmpty() const \{ \\
\codeindent{2}... \\
\}\\
 \\
\#endif /* ! MYAPP\_VECTOR\_HXX */
\end{algorithm}

\subsection{Inline functions}
\label{sec::inline}
Inline functions must follow the same rules that templates
(section~\ref{sec::templates}).
