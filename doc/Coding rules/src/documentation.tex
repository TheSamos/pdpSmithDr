\section{Documentation}

\subsection{Code comments}
Comments must be in English.\\
You may write a short description of your classes and member
functions. Describing arguments is really useful everytime you can
raise a critical question, e.g. ``Who owns the data? Who is
responsible for memory free?''. Don't be shy, tell other people.\\

If an algorithm is described in a paper, it is a good idea to add a
reference to a paper, or better an equation number in a paper, or
explanations of differences with the paper.\\
It is recommanded to use Doxygen to document your code.

\subsection{Gotchas}
Embed special keywords in your code comments so that an automatic
process can establish a maintenance report. There is a list of these
gotchas:
\begin{itemize}
\item[$\bullet$] {\bf $:$TODO$:$ topic} \\ Means there is more to do
  there, don't forget.
\item[$\bullet$] {\bf $:$BUG$:$ [bugid] topic} \\ Means there is a {\it
  known} bug here; explain it and optionnaly give a bug id.
\item[$\bullet$] {\bf $:$GLITCH$:$} \\ Means there is a {\it suspected}
  bug here, need more debug to confirm.
\item[$\bullet$] {\bf $:$COMMENT$:$} \\ Add a specific note you want to be
  reported to others, e.g. to raise a warning.
\item[$\bullet$] {\bf $:$TRICKY$:$} \\ Tells somebody that the following
  code is very tricky so don't go changing it without thinking.
\item[$\bullet$] {\bf $:$KLUDGE$:$} \\ When you've done something ugly (Do
  you? Really?), say so and explain how you would do it differently
  next time if you had more time.
\item[$\bullet$] {\bf $:$OPTIM$:$} \\ An optimization may be possible
  or needed here.
\item[$\bullet$] {\bf $:$PARALLEL$:$} \\ Comment relative to possible
  parallelization.
\end{itemize}
\ \\
Gotchas comment may consist of several lines, but the first line should be a self-containing, meaningful summary, respecting the following line format:
\begin{center}
  {\bf $:$GOTCHA$:$ author date$:$ summary}
\end{center}
Date must be in the form {\it yymmdd}.\\

For example:
\begin{algorithm}[H]
// $:$TODO$:$ tom 091127$:$ implement case of missing data. \\
// We should read incomplete datasets and fill missing parts \\
// with a predefined default value.
\end{algorithm}

\subsection{Project documentation}
It is mandatory to have a {\bf README.txt} file at the root of your
project that explains how to compile your code (and thus on which
other libraries/programs it depends), how to execute your code and
what your code does. You may also add the publications where
algorithms are described.\\ It is very useful to have an explanation
of all the parameters and the range of validity for their value. It is
also very interesting to know which parameters must be modified first
to improve the results.\\ It is a good practice to have good default
values for all parameters, and a simple help that may be displayed
when the program is run (a {\bf -h} option for example).
