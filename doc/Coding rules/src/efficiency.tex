\section{Code efficiency}

% functions
\subsection{Functions}
\subsubsection{Use standard functions.}
Use {\bf STL} facilities (vector, list, string, iostream,
stringstream, and algorithms) or {\bf standard functions} (memcpy,
memset, ..) whenever possible.\\
Most often, these functions are more optimized than what you could
ever write.\\

For example, replace:
\begin{algorithm}[H]
int arr[100]; \\
memset(arr, 0, sizeof(int)*100); \\
f(arr);
\end{algorithm}

or
\begin{algorithm}[H]
int* arr = new int[100]; \\
memset(arr, 0, sizeof(int)*100); \\
f(arr); \\
delete [] arr;
\end{algorithm}

by
\begin{algorithm}[H]
const size\_t size = 100; \\
std$::$vector<int> arr(size, 0); \\
f(\&arr[0]);
\end{algorithm}

For example, replace:
\begin{algorithm}[H]
  char str[100];
\end{algorithm}

by:
\begin{algorithm}[H]
std$::$string str;
\end{algorithm}

For example, replace:
\begin{algorithm}[H]
int a, b; \\
char line[500]; \\
... \\
sscanf(line, "\%d \%d", \&a, \&b); 
\end{algorithm}

by
\begin{algorithm}[H]
int a, b; \\
std$::$string line; \\
... \\
std$::$stringstream ss(line); \\
ss $>>$ a; \\
ss $>>$ b;
\end{algorithm}

\subsubsection{Const correctness.}
Use {\bf const} keyword after each function of a class that does not
modify members of the class, for each pointer or reference parameter
of a function when applicable, and even for unmodified local
variables. It greatly helps to understand the code.\\

For example:
\begin{algorithm}[H]
MyClass::MyClass(const Matrix\& m) : m\_mat(m) \{\} \\
 \\
const Matrix\& \\
MyClass::getMatrix() const \{ \\
\codeindent{1}return m\_mat; \\
\}
\end{algorithm}

\subsubsection{Initialization list.}
Use initialization list for constructor (better performance wise).\\
In particular, inialize pointers to NULL in initialization lists.\\ 

For example, replace:
\begin{algorithm}[H]
MyClass::MyClass(int a, int b, int c) \{ \\
\codeindent{1}m\_a = a; \\
\codeindent{1}m\_b = b; \\
\codeindent{1}m\_c = c; \\
\codeindent{1}m\_d = NULL; \\
\}
\end{algorithm}

by:
\begin{algorithm}[H]
MyClass::MyClass(int a, int b, int c) \\
\codeindent{1}: m\_a(a), m\_b(b), m\_c(c), m\_d(NULL) \{\}
\end{algorithm}

% variables
\subsection{Variables}
\subsubsection{Reduce variable scope.}
It is C++, not C, so variables do not need to be declared at the start
of the block.\\
Hence, declare variables near their use, then the code is easier to
read, and the compiler can often do a better job.\\

For example:
\begin{algorithm}[H]
\codeindent{1}for (int i=0; i<N; ++i) \{ \\
\codeindent{2}const int g = fct(i); \\
\codeindent{2}... \\
\codeindent{1}\}
\end{algorithm}

However, avoid local declarations that hide declarations at higher
levels. For example, do not declare the same variable name in an inner
block.\\

For example:
\begin{algorithm}[H]
\codeindent{1}int count; \\
\codeindent{1}... \\
\codeindent{1}int f(...) \{ \\
 \\
\codeindent{2}if (condition) \{ \\
\codeindent{3}int count; /* AVOID! */ \\
\codeindent{3}... \\
\codeindent{2}\} \\
\codeindent{2}... \\
\codeindent{1}\}
\end{algorithm}

\subsubsection{Incrementation.}
Prefer pre-increment to post-increment when they are equivalent.\\
Some compiler produce a useless temporary when post-increment is used
(on complex types), thus it is better, performance-wise, to use
pre-increment.\\

For example:
\begin{algorithm}[H]
\codeindent{1}typedef std$::$vector<int> VectorInt; \\
\codeindent{1}VectorInt v; \\
\codeindent{1}... \\
\codeindent{1}for (VectorInt$::$const\_iterator it = v.begin(); it != v.end(); ++it) \{ \\
\codeindent{2}... \\
\codeindent{1}\}
\end{algorithm}

\subsubsection{Dynamic allocations.}
Avoid dynamic allocations (new/malloc) as much as possible, in
particular in performance critical code.\\
For example, for complex types (requiring an allocation for example)
it could be better, performance-wise, to declare them outside of loops
(and thus not the nearest to their use).\\

For example, replace:
\begin{algorithm}[H]
\codeindent{1}for (int i=0; i<N; ++i) \{ \\
\codeindent{2}Matrix m(nbRows, nbCols); \\
\codeindent{2}... \\
\codeindent{1}\}
\end{algorithm}

by:
\begin{algorithm}[H]
\codeindent{1}Matrix m(nbRows, nbCols); \\
\codeindent{1}for (int i=0; i<N; ++i) \{ \\
\codeindent{2}... \\
\codeindent{1}\}
\end{algorithm}

% others
\subsection{Forward declaration}
Forward declarations should be employed when safely applicable to
minimize \emph{\#include} dependencies and reduce compilation time.\\
Forward declarations may be applied to avoid including declarations of
objects that are used solely by pointer/reference or as function
return values; they may not be used if the object serves as a base
class, or as a non-pointer/reference class member or parameter.\\

For example, replace:
\begin{algorithm}[H]
\#include "Matrix.hpp" \\
\#include "Image.hpp" \\
 \\
matrixToImage(const Matrix\& m, Image\& img);
\end{algorithm}

by:
\begin{algorithm}[H]
class Matrix; \\
class Image; \\
 \\
matrixToImage(const Matrix\& m, Image\& img);
\end{algorithm}

Here, includes of Matrix.hpp and Image.hpp will only be done in .cpp file.
